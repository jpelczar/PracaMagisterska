\section{Wstęp}
Tematem niniejszej pracy jest "Optymalizacja wykorzystania materiału w procesie rozkroju rur". W skład rozwiązania omawianego tytułu wchodzi rozwiązanie problemu optymalnego rozkroju (ang. "Cutting Stock Problem"). Uogólnionym przypadkiem tego zagadnienia jest problem plecakowy (ang. "Knapsack Problem").

Motywacją niniejszej pracy jest szerokie zastosowanie problemu optymalnego rozkroju w przemyśle. Rozkrój rur jest problemem jednowymiarowym wykorzystywanym między innymi w procesie wyrobu krzeseł w trakcie podziału elementów na podstawę krzesła. Typowym przykładem procesu rozkroju jednowymiarowgo jest podział rolki papieru w przemyśle papierniczym. Dwuwymiarowy problem rozkroju jest wykorzystywany w trakcie wykroju elementów z arkuszów blachy. Uogólnienie zagadnienia do problemu plecakowego znajduje zastosowanie w wielu dziedzinach życia takich jak transport oraz kryptografia.

Celem niniejszej pracy jest stworzenie programu obliczającego problem optymalnego rozkroju dwiema metodami: brutalnej siły oraz opóźnionej generacji kolumn. Intencją opracowania tematu jest również porównanie obu metod i określenie ich przydatności w procesie technologicznym.

\hyperref[sec:knapsack]{Rozdział pierwszy} wprowadza temat problemu plecakowego. Opisuje on jego zastosowanie oraz podaje przykładowe rozwiązanie. Temat ten jest istontny w kontekście problemu optymalnego rozkroju, gdyż jest to ogólny problemu optymalnego rozkroju.

\hyperref[sec:cuttingStockProblem]{Rozdział drugi} opisuje problem optymalnego rozkroju. Opisuje on podstawy teoretyczne oraz zastosowanie praktyczne zadanego problemu. Jest on podzielony na podrozdziały które kolejno opisują rozwiązanie problemu metodą opóźnionej generacji kolumn oraz brutalnej siły. Metody te znacznie się od siebie różnią. Pierwsza z nich bazuje na obliczeniach matematycznych, natomiast druga na intuicyjnej definicji problemu.

\hyperref[sec:implementation]{Rozdział trzeci} opisuje sposób implementacji programu użytego do generacji statystyk porównania algorytmów oraz programu możliwego do wykorzystania przez użytkownika w celu otrzymania schematu rozkroju zadanym algorytmem.

\hyperref[sec:result]{Rozdział czwarty} opisuje wyniki uzyskane eksperymentalnie z porównania wykonania obu algorytmów. Rozdział ten zawiera również przykładowe rozwiązania uzyskane w trakcie przeprowadzania eksperymentu.
