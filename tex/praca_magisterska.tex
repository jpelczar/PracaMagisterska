\documentclass[11pt,a4paper]{article}

\usepackage[polski]{babel}
\usepackage[T1]{polski}
\usepackage[utf8]{inputenc}
% \usepackage{fontspec}
\usepackage{amsmath}
\usepackage{hyperref}
\usepackage{graphicx}
\usepackage{algorithm}
\usepackage[noend]{algpseudocode}
\usepackage{tikz}
\usepackage{wrapfig}
\usepackage{csvsimple}

\graphicspath{ {figures/} }
\numberwithin{equation}{section}
\numberwithin{figure}{section}
\makeatletter
\crefname
\renewcommand{\ALG@name}{Algorytm}
\hypersetup{
    colorlinks=true,
    linktoc=all,
    citecolor=black,
    filecolor=black,
    linkcolor=black,
    urlcolor=black
}
\graphicspath{ {../image} }
\setcounter{secnumdepth}{5}
\makeatletter
\def\old@comma{,}
\catcode`\,=13
\def,{%
  \ifmmode%
    \old@comma\discretionary{}{}{}%
  \else%
    \old@comma%
  \fi%
}
\makeatother

\title{Optymalizacja wykorzystania materiału w procesie rozkroju rur}
\author{Jakub Pelczar}
\date{\today\\v0.3}

\begin{document}

\maketitle
\tableofcontents

\section{Wstęp}
Tematem niniejszej pracy jest "Optymalizacja wykorzystania materiału w procesie rozkroju rur". W skład rozwiązania omawianego tytułu wchodzi rozwiązanie problemu optymalnego rozkroju (ang. "Cutting Stock Problem"). Uogólnionym przypadkiem tego zagadnienia jest problem plecakowy (ang. "Knapsack Problem").

Motywacją niniejszej pracy jest szerokie zastosowanie problemu optymalnego rozkroju w przemyśle. Rozkrój rur jest problemem jednowymiarowym wykorzystywanym między innymi w procesie wyrobu krzeseł w trakcie podziału elementów na podstawę krzesła. Typowym przykładem procesu rozkroju jednowymiarowgo jest podział rolki papieru w przemyśle papierniczym. Dwuwymiarowy problem rozkroju jest wykorzystywany w trakcie wykroju elementów z arkuszów blachy. Uogólnienie zagadnienia do problemu plecakowego znajduje zastosowanie w wielu dziedzinach życia takich jak transport oraz kryptografia.

Celem niniejszej pracy jest stworzenie programu obliczającego problem optymalnego rozkroju dwiema metodami: brutalnej siły oraz opóźnionej generacji kolumn. Intencją opracowania tematu jest również porównanie obu metod i określenie ich przydatności w procesie technologicznym.

\hyperref[sec:knapsack]{Rozdział pierwszy} wprowadza temat problemu plecakowego. Opisuje on jego zastosowanie oraz podaje przykładowe rozwiązanie. Temat ten jest istontny w kontekście problemu optymalnego rozkroju, gdyż jest to ogólny problemu optymalnego rozkroju.

\hyperref[sec:cuttingStockProblem]{Rozdział drugi} opisuje problem optymalnego rozkroju. Opisuje on podstawy teoretyczne oraz zastosowanie praktyczne zadanego problemu. Jest on podzielony na podrozdziały które kolejno opisują rozwiązanie problemu metodą opóźnionej generacji kolumn oraz brutalnej siły. Metody te znacznie się od siebie różnią. Pierwsza z nich bazuje na obliczeniach matematycznych, natomiast druga na intuicyjnej definicji problemu.

\hyperref[sec:implementation]{Rozdział trzeci} opisuje sposób implementacji programu użytego do generacji statystyk porównania algorytmów oraz programu możliwego do wykorzystania przez użytkownika w celu otrzymania schematu rozkroju zadanym algorytmem.

\hyperref[sec:result]{Rozdział czwarty} opisuje wyniki uzyskane eksperymentalnie z porównania wykonania obu algorytmów. Rozdział ten zawiera również przykładowe rozwiązania uzyskane w trakcie przeprowadzania eksperymentu.

\section{Knapsack Problem - Problem plecakowy}
Problem plecakowy jest zagadnieniem optymailzacyjnym. Problem ten swoją nazwę wziął z analogii do rzeczywistego problemu pakowania plecaka. Rozwiązując ten problem zarówno w praktyce jak i teorii trzeba zachować reguły określające ładowność plecaka dotyczące objętości i nośności plecaka. Knapsack Problem zaczął być intensywnie badany po pionierskiej pracy Dantziga \cite{DantzigArticle} w późnych latach 50 XX wieku. Znalazł on natychmiast zastosowanie w przemyśle oraz w zarządzaniu finansami. Z teoretycznego punktu widzenia, problem plecakowy często występuję jako relaksacja róznorodnych problemów programowania całkowitego \cite{PisingerThesis}.
\subsection{Różnorodność problemu plecakowego}
Wszystkie elementy z rodziny tego problemu wymagają pewnego zestawu elementów które mogą zostać wybrane w taki sposób że zysk zostanie zmaksymalizowany, a pojemość placaka lub plecaków nie zostanie przekroczona. Wszystkie typy problemu należą do rodziny problemów $NP-trudnych$ co oznacza, że raczej nispotykane jest rozwiązanie problemu z użyciem algorytmów wielomianowych. Możliwe są różne warinaty problemu zależna od rozmieszczenia elementów oraz plecaków:
\begin{itemize}
  \item \textit{Problem plecakowy 0-1} - każdy element może być wybrany tylko raz. Problem polega na wyborze $n$ elementów dla których suma profitów $p_j$ jest największa, bez konieczności osiągnięcia całkowitej pojemności $c$. Może być sformułowany jako problem maksymalizacji:
  \begin{equation}
    \begin{aligned}\label{01knapsack}
      & \textrm{maksymalizacja} & & \sum_{j=1}^n p_jx_j, \\
      & \textrm{w odniesieniu do} & & \sum_{j=1}^n w_jx_j \le c, \\
      &&& x_j \in \{0,1\},\quad j = 1,\dots,n \\
    \end{aligned}
  \end{equation}
  \item \textit{Ograniczony problem plecakowy} - każdy element może być wybrany ograniczoną ilość razy.
  \item \textit{Problem plecakowy wielokrotnego wyboru} - elementy powinny być wybierane z klas rozłącznych.
  \item \textit{Wielokrotny problem plecakowy} - wiele plecaków jest wypełnianych równocześnie.
  \item \textit{Welokrotnie ograniczony problem plecakowy} - najbardziej ogólny typ który jest problemem programowania całkowitego z dodatnimi współczynnikami.
\end{itemize}

\subsection{Możliwe rozwiązania}

Dopóki problem plecakowy należy do problemów $NP-trudnych$ nie jest znane inne dokładne rozwiązanie niż wyliczenie przestrzeni rozwiązań. Użycie poniższych technik może ograniczyć pracochłonność otrzymania rozwiązania:
\begin{itemize}
  \item \textit{Metoda podziału i ograniczeń} - pełna enumeracja rozwiązań, ale ograniczenia są użyte do znalezienia węzłów które nie mogą doprowadzić do poprawy rozwiązania. Metoda ta często jest stosowana do problemu plecakowego od momentu gdy Kolesar \cite{KolesarArticle} zaprezentował pierwszy algorytm w 1967 roku.
  \item \textit{Programownaie dynamiczne} - może być traktowane jako enumeracja wszerz z pewnymi zasadami dominacji. Czasem testy brzegowe są dodawane do algorytmu programowania dynamicznego, wtedy algorytm ten staje się "zaawansowaną" formą metody podziału i ograniczeń.
  \item \textit{Przestrzeń stanów relaksacji} - jest to relaksacja metody programowania dynamicznego w której współczynniki są skalowane przez ustaloną wartość. Dzięki tej metodzie zmniejsza się czas oraz złożoność algorytmu, ale rozwiązanie traci optymalność. Algorytm ten jest często wykorzystywany jako wydajny algorytm aproksymacji problemu plecakowego.
  \item \textit{Przetwarzanie wstępne} - pewna liczba zmiennych zostaje ustalona jako wartość optymalana, używając testów brzegowych do wykluczenia pewnych wartości z rozwiązania.
\end{itemize}

\section{Cutting Stock Problem - Problem optymalnego rozkroju}
\label{sec:cuttingStockProblem}

% \section{Metoda "Brutal Force"}
\subsection{Algorytm wyjściowy}
Metoda ta opiera się zarówno na intuicji jak i na rozwiązaniu zaproponowanym przez Dantziga dla problemu plecakowego \cite{DantzigArticle}. Jest to metoda która w prosty sposób - nie używając złożonych modeli matematycznych, pozwala osiągnąć optymalny rozkrój materiału.

Pierwszym krokiem jest posortowanie malejąco po długości elementów wyściowych. $l_1 \ge l_2 \ge ... \ge l_m$

Drugim krokiem jest pobranie pierwszego elementu z kolejki i sprawdzenie, jak wiele razy dana długość zawiera się w długości elementu bazowego. Obliczone zostaje ile materiału pozostało w elemencie bazowym. Pobierany jest następny odcinek z kolejki. Zostaje sprawdzone ile razy zawiera się w pozostałej długości.
\begin{equation}\label{base_dantizg}
a_1 = [L/l_1], a_2 = [(L-l_1*a_1)/l_2], a_3 = [(L-(l_1*a_1+l_2*a_2))/l_3], ...
\end{equation}
Kroki te powtarzane są dopóki kolejka się nie skończy.

Każdy element wyjściowy posiada określoną liczebność jaką powinien osiągnąć na końcu procesu. Jeśli licznik jest równy zeru wówczas długość jest pomijana. Koniecznie jest sprawdzenie czy otrzymany wynik jest mniejszy lub równy od wymaganej ilości:
\begin{itemize}
  \item Jeśli stwierdzenie jest prawdziwe - długość z której elementy są wycinane zostanie zmniejszona o liczbę wystąpień wykrojów w aktywności (zestawie elementów wykroju) pomnożoną przez długość elementu, a licznik wymaganych odcinków danej długości zostanie zmniejszony o odpowiednią liczbę wystąpień
  \item Jeśli stwierdzenie jest fałszywe - długość z której elementy są wycinane zostanie zmniejszona o liczbę dostępnych wykrojów pomnożoną przez długość elementu, a licznik wymaganych odcinków danej długości zostanie ustawiony na zero.
\end{itemize}
Po zakończeniu przebiegu algorytmu dla danego układu wykrojów określa się ile razy dana aktywność może zostać użyta. Można to wyznaczyć poprzez obliczenie $g = min\{z_i/a_i\}, i \in {0,..,m}$, gdzie $z$ to pozostała ilość wykrojów elementu $i$, $a$ to ilość wykrojów elementu $i$ w danej aktywności. Następnie zmniejsza się o $g$ licznik dostępnych odcinków danego elementu dla którego $a_i > 0$.

Cały proces powtarzany jest do momentu aż wszytskie wymagane elementy zostaną wycięte.

\subsection{Rozszerzenie o szerokość cięcia}
W warunkach rzeczywistych elementy wycinane są za pomocą ostrza które ma niezerową szerokość. Wówczas metodę obliczania należy rozszerzyć jeśli ma odpowiadać warunkom rzeczywistym. Szerokość cięcia wlicza się w odpad. Jest kilka przypadków wliczania szerokości ostrza.

Jeżeli element jest równy długości bazowej wówczas nie wlicza się szerokości cięcia. Natomiast jeżeli materiał bazowy ma zostać pocięty na kilka elmentów wówczas do każdego dolicza się szerokość cięcia. Szczególnym przypadkiem jest, gdy ostatni element wraz z szerokością ostrza jest dłuższy niż długość odcinka, który został po wycięciu wcześniejszych elementów.

Gdyby szerokość cięcia nie zostałą uwzględniona w obliczeniach wówczas dla elementu wejściowego o długości 6000mm i wymaganych odcinkach 4500mm oraz 1500mm, obie długości zostały wycięte z jednego segmentu materiału bazowego. Skutkiem takiego postępowania byłby element krótszy o szerokość ostrza. Zazwyczaj dłuŋość ta może być akceptowana jako toleracncja dokładności maszyny. Jednak dla poprawności obliczeń wielkość ta powinna zostać uwzględniona.

\subsection{Rozszerzenie o wiele długości bazowych}
Dla zmniejszenia odpadu można użyć kilku długości bazowych. Rozszerzenie to wprowadza następująca zmianę algorytmu: obliczenia układu muszą zostać powtórzone dla każdego elementu wejściowego. Następnie wybierany jest ten rozkrój, który daje mniejszy odpad. Modyfikacja ta znacząco wpływa na wydajność metody. Jeżeli $n$ oznacza złożoność obliczeniową podstawowego algorytmu, a $m$ oznacza liczbę odcinków wejściowych, wówczas nowa złożonośc obliczeniowa wynosi $m*n$.

\subsection{Rozszerzenie o cenę materiału wsadowego}
Rozszerzenie to wprowadza zmianę koncepcyjną. Każdy element bazowy posiada cenę za metr bieżący materiału, umożliwia to obliczenie kosztu odpadu i wybranie tańszej opcji wykroju.

\subsection{Przykład}
\begin{enumerate}
  \item Dane wejściowe
  \begin{itemize}
    \item 6000mm - 3\$/mb
    \item 7000mm - 2\$/mb
    \item szerokość cięcia: 10mm
  \end{itemize}
  \item Dane wyjściowe
  \begin{itemize}
    \item 1x3500mm
    \item 1x3000mm
    \item 3x2000mm
    \item 5x500mm
  \end{itemize}
  \item Przebieg algorytmu
  \begin{itemize}
    \item Pierwszy rozkrój
    \begin{itemize}
      \item 3500mm mieści się raz w 6000mm. Zostaje $2500 - 10 = 2490$mm.
      \item 3000mm nie mieści się w 2490mm.
      \item 2000mm mieści się raz w 2490mm. Zostaje $490 - 10 = 480$mm.
      \item 500mm nie mieści się w 480mm.
      \item Rozkrój 6000mm: 3500mm, 2000mm. Odpad $6000 - 5500 = 500 * 0.003 = 1.5\$$
      \item ------------
      \item 3500mm mieści się dwa razy w 7000mm. Dostępny jest jeden odcinek 3500mm. Zostaje $3500 - 10 = 3490$mm.
      \item 3000mm mieści sie raz w 3490mm. Zostaje $490 - 10 = 480$mm.
      \item 2000mm nie mieści się w 480mm.
      \item 500mm nie mieści się w 480mm.
      \item Rozkrój 7000mm: 3500mm, 3000mm. Odpad $7000 - 6500 = 500 * 0.002 = 1.0\$$
      \item ------------
      \item Wybrano rozkrój 3500mm, 2000mm na długości 7000mm ze względu na mniejszy koszt odpadu.
      \item 0x3500mm; 0x3000mm; 3x2000mm; 5x500mm
    \end{itemize}
    \item Drugi rozkrój
    \begin{itemize}
      \item 2000mm mieści się trzy razy w 6000mm. Uwzględniając szerokość cięcia - zostaną użyte tylko dwa elementy od długości 2000mm. Zostaje $2000 - 2*10 = 1980$mm.
      \item 500mm mieści się trzy razy w 1980mm. Zostaje $480 - 3*10 = 450$mm.
      \item Rozkrój 6000mm: 2x2000mm, 3x500mm. Odpad $6000 - 5500 = 500 *0.003 = 1.5\$$
      \item ------------
      \item 2000mm mieści się trzy razy w 7000mm. Zostaje $1000 - 3*10 = 970$mm.
      \item 500mm mieści się raz w 970mm. Zostaje $470 - 10 = 460$mm.
      \item Rozkój 7000mm: 3x2000mm, 500mm. Odpad $7000 - 6500 = 500 * 0.002 = 1.0\$$
      \item ------------
      \item Wybrano rozkrój 3x2000mm, 500mm na długości 7000mm ze względu na mniejszy koszt odpadu
      \item 0x3500mm, 0x3000mm, 0x2000mm, 4x500mm
    \end{itemize}
    \item Trzeci rozkrój
    \begin{itemize}
      \item 500mm mieści się dwanaście razy w 6000mm. Dostępne są cztery element 500mm. Zostaje $6000 - 4*500 - 4*10 = 3960$mm.
      \item Rozkrój 6000mm: 4x500mm. Odpad $6000 - 4*500 = 4000 * 0.003 = 12\$$
      \item ------------
      \item 500mm mieści się czternaście razy w 7000mm. Dostępne są cztery elementy 500mm. zostaje $7000 - 4*500 - 4*10 = 4960$mm
      \item Rozkrój 7000mm: 4x500mm. Odpad $7000 - 4*500 = 5000 * 0.002 = 10\$$
      \item ------------
      \item Wybrano rozkrój 4x500 na długości 7000mm ze względu na mniejszy koszt odpadu
      \item 0x3500mm, 0x3000mm, 0x2000mm, 0x500mm
    \end{itemize}
    \item Podsumowanie
    \begin{itemize}
      \item Rozkroje : 3500mm, 2000mm na długości 7000mm; 3x2000mm, 500mm na długości 7000mm; 4x500 na długości 7000mm.
      \item Suma odpadów: $6000 * 0.002 = 12\$$
    \end{itemize}
  \end{itemize}
\end{enumerate}

\subsection{Podsumowanie}
Przedstawiony algorytm jest intuicyjny oraz zwraca poprawne wyniki. Główną wadą jest brak świadomości o następnym kroku oraz kolejnych wykrojach. Dla przykładu: Zosatło 1000mm materiału, do dyspozycji (z długości mniejszych niż 1000mm) jest odcinek 900mm oraz dwa elementy 480mm. Algorytm przydzieli odcinek 900mm, jednak lepszym wyborem byłoby użycie dwóch odcinków 480mm.

% \newcommand{\mf}[1]{\textbf{\textit{#1}}}
\newcommand{\tsub}[1]{\textsubscript{#1}}
\newcommand{\tsuper}[1]{\textsuperscript{#1}}

\section{Metoda "Delayed Column Generation"}

\subsection{Algorytm}

\begin{equation}\label{length_eq}
L \ge l_1a_1+\dots+l_ma_m
\end{equation}

\begin{equation}\label{cost_eq}
b_1a_1+\dots+b_ma_m > c
\end{equation}

\begin{enumerate}
\item Określnie $m$ poczatkowych rokrojów i ich kosztu w następujący sposób: dla każdego $i$ wybranie długości bazowej $L_j$ dla której $L_j > l_i$ i określenie $i$-tego rokroju jako wycięcie $a_{ii} = [L_j / l_i]$ elementów o długości $l_i$ z długości $L_j$. Koszt $i$-tego rozkroju będzie równy kosztowi $c_j$ długości $L_j$ z której $i$-ta operacja wycina odcinki o długości $l_i$.
\item Uformowanie macierzy \mf{B}
\[
\begin{matrix}
1 & -c_1  & -c_2  & \dots & -c_m \\
0 & a_{11}  & 0 & \dots & 0 \\
0 & 0 & a_{22}  & \dots & 0 \\
\vdots&\vdots&\vdots&\ddots&\vdots \\
0 & 0 & 0 & \dots & a_{mm}
\end{matrix}
\]
gdzie $a_{ii}$ jest ilością odcinków o długości $l_i$ wyciętych w $i$-tym rozkroju z długości bazowej o koszcie $c_j$. Ostatnie $m$ kolumn jets powiązane z rozkrojami. Dane te będą aktualizowane gdy zostanie znaleziony wynik który poprawi rozwiązanie.

Utworzenie $m$ $m+1$ wymiarowych wektorów kolumnowych \mf{S\tsub{1}},...,\mf{S\tsub{m}} odnoszących się do zmiennych dodatkowych, gdzie \mf{S\tsub{i}} posiada same zera z wyjątkiem wiersza $(i+1)$ w którym jest $-1$. Dodatkowo utworzenie $m+1$ wymiarowego wektora kolumnowego \mf{N'} który jako pierwszy element przyjmuje 0, a w następnych $i$-tych wierszach posiada wartośic $N_i$.

Obliczenie \mf{B}\tsuper{-1} która wynosi:
\[
\begin{matrix}
1 & c_1/a_{11}  & c_2/a_{22}  & \dots & c_m/a_{mm} \\
0 & 1/a_{11}  & 0 & \dots & 0 \\
0 & 0 & 1/a_{22}  & \dots & 0 \\
\vdots&\vdots&\vdots&\ddots&\vdots \\
0 & 0 & 0 & \dots & 1/a_{mm}
\end{matrix}
\]

Niech $\mf{N} = \mf{B}^{-1} \cdotp \mf{N'}$. Sprawdzając czy pierwszy element z $\mf{B}\tsuper{-1} \cdotp \mf{P}$ jest dodatni można określić czy istnieje możliwość polepszenia rozwiązania. Wektor kolumnowy \mf{P} jets wektorem złożonym ze zmiennych nieuzytych w bieżącym rozwiązaniu, np. pierwszy element to negatywny koszt, a pozostałe $m$ wierszy jest równe zmiennym $a_{ij}$.

\item Z powyższego puntku wynika że jeśli $i$-ta zmienna dodatkowa która nie wchodzi w skład rozwiązania może ulepszyć rozwiązanie wtedy i tylko wtedy gdy $(i+1)$ element pierwszego wiersza \mf{B}\tsuper{-1} jest ujemny.

\item Jeśli nie jest możliwe polepszenie rozwiązania nalezy określić czy wprowadznie nowego rozkroju ulepszy bieżące rozwiązanie. Jets to możliwe poprzez sprawdznie czy dla $L$ z kosztem $c$ istnieje rozwiązanie nierówności \ref{length_eq} oraz \ref{cost_eq}, gdzie $b_1,\dots,b_m$ to ostatnie $m$ elementów w piwerwszym wierszu \mf{B}\tsuper{-1}. Jeśli te nierównoście nie posiadają rozwiąania dla dowolnej długości $L_1,\dots,L_k$ z kosztem odpowiednio $c_1,\dots,c_m$ wtedy bieżące rozwiązanie jest minimum. Rozwiązanie i jego koszt jest określone poprzez \mf{N}, gdzie pierwszy wiersz to koszt, a pozostałe $m$ wierszy jest, w kolejności, odpowiednimi wartościami $m$-tej kolumny z \mf{B}\tsuper{-1}.

Jeśli nowy rozkrój poprawi rozwiązanie wtedy formowany jest nowy wektor \mf{P} ze współczynnikami, w kolejności $-c,a_1,a_2,\dots,a_m$.

\item Wprowadznie zarówno dodatkowej zmiennej jak i nowego rozkroju może poprawić rozwiązanie. W obu przypadkach \mf{P} będzie kolumnowym wektorem ze zmiennymi. Dla określenia nowych \mf{B}\tsuper{-1} oraz \mf{N} które opisują ulepszone rozwiązanie i jego koszt, co pozawala na przejście przez kkroki 3, 4 oraz kontynujacje kroku 5 w nastepujący sposób: Obliczenie $\mf{B}\tsuper{-1} \cdotp \mf{P}$ - niech elementy wynikime będą elementy $y_1,\dots,y_m,y_{m+1}$ oraz niech elementami bierzącego wektora \mf{N} będą $x_1,\dots,x_m,x_{m+1}$. Ustalenie $i$, $ i \ge 2$ dla każdego $y_i > 0$, $x_i \ge 0$ oraz $x_i/y_i$ jest najminiejsze i przypisanie tej wartości do zmiennej $k$. Minimalny stosunek powinien być zerem aby można było wykorzystać metodę degeneracji.

Jeśli stosunek nie jest równy zero wtedy $k$-ty element wektora \mf{P}, $y_k$ będzie elementem wokół którego zajdzie eliminacja Gaussa odbywająca się równocześnie na \mf{B}\tsuper{-1}, $\mf{B}\tsuper{-1} \cdotp \mf{P}$ oraz \mf{N}. Eliminacja ta przebiega na macierzy $(m+1) \times (m+3)$ wymiarowej \mf{G} uformowanej z \mf{B}\tsuper{-1} poprzez dołączenie kolumn $\mf{B}\tsuper{-1} \cdotp \mf{P}$ oraz \mf{N}. Pierwsze $m+1$ kolumn \mf{G'} formuje nową macierz \mf{B}\tsuper{-1}, a kolumna $m+2$ to nowy wektor \mf{N}. Zależność między kolumnami \mf{B}\tsuper{-1} a rozkrojami lub zmiennymi dodatkowymi jets aktualizowana poprzez usunięcie $k$-tej kolumny i podmienieniu jej na nowy rozkruj lub zmienną dodatkową.

Degeneracja w razie wystąpienia może być obsłużona w tradycyjny sposób. Pewne środki ostrożności powinny zostać podjęte w celu uniknięcia cykliczności. Nowa kolumna \mf{N}\tsuper{1} z dodatnimi elementami $x_1',\dots,x_{m+1}'$ która jest niezależna pd \mf{N} jest dołączana do \mf{G} i wybór takiego $y_i > 0$ dla którego $x_i = 0$ który jest elementem osiowym jets dokonywany na podstawie takiego $i$ dla którego $x_i' > 0$ oraz $x_i'/y_i$ jest najmniejsze. Gdy element osiowy zostanie wybrany, wówczas eliminacja Gaussa zachodzi tak jak w poprzednim przypadku na powiększonej macierzy \mf{G}. Dodatkowa kolumna jest przechowywane w \mf{G} dopóki istnieje takie $i$ dla którego $x_i/y_i$ jest dodatnie i skończone, jeśłi warunek ten jest spełniony wówczas kolumna zostaje usunięta. Powinno to nastąpić w przypadku gdy nie istnieje takie $i$ dla którego $x_i/y_i$ oraz $x_i'/y_i$ są dodatnie i skończone. Wówczas powinna zostać dodana kolumna \mf{N}\tsuper{2} nizależna od \mf{N} oraz \mf{N}\tsuper{1}. Podobnie dowolna liczba kolumn może zostac dodana i usunięta gdy przestanie byc potrzebna. Dopóki kolumny są niezależne w czasie dodawania i pozostają takie po eliminacji Gaussa, nie potrzeba więcej jak $m$ nowych kolumn. Każda dodana kolumna definiuje nowy problem liniowy który eliminuje problem cykliczności tak długo aż degeneracja nie wystąpi.

\end{enumerate}

\subsection{Metody użyte w implementacji}

\paragraph{Dwufazowa metoda simplex}
\paragraph{Metoda podziału i ograniczeń}

\subsection{Przykład}

\section{Wyniki}
\subsection{Porównanie}
% \begin{table}
%   \centering
%   \caption{Test table}
%   \label{test_table}
%   \csvautotabular[separator=semicolon]{../csv/time.csv}
% \end{table}
\subsection{Wnioski}
\subsection{Podsumowanie}

\section{Opis implementacji}
Aplikacja została napisana przy użyciu języków programowania bazujących na maszynie wirtualnej javy:
\begin{enumerate}
  \item Kotlin - podstawowy język użtyty do implementacji (prawie 90\% projektu)
  \item Java 8 - język użyty do generowania statystyk wykonania algorytmów
  \item JavaFX - technologia zastosowana do stworzenia graficzego interfejsu użytkownika (wraz z CSS)
\end{enumerate}

Architektura aplikacji jest modułowa. Zostały wydzielone części odpowiedzialne za obliczenia matematyczne, implementacje algorytmów, zapis i odczyt plików CSV, generację statystyk, definicję modelu danych oraz moduł zawierający aplikację korzystająca z pozostałych pakietów (\hyperref[fig:arch]{rys.~\ref*{fig:arch}}).

\begin{figure}[h]
  \center
  \includegraphics[scale=0.7]{../image/arch.png}
  \caption{Architektura aplikacji}
  \label{fig:arch}
\end{figure}

Architektura modułu odpowiedzialnego za implementację algorytmów posiada strukturę wzorca projektowego fasada. Klasy odpowiedzialne za konkretną implementację metody obliczenia rozkrojów rozszerzają klasę abstarkacyjną która definiuje wspólne funkcje oraz deklaruje metody które powinny zostać zdefiniowane w klasach potomnych. Główny moduł aplikacji wraz z modułem odpowiedizlanym za model danych tworzy implementację wzoraca Model-Widok-Kontroler (MVC). Klasa kontrolera zarządza widokiem stworzonym w języku FXML.

Rysunki \ref{fig:empty_win} oraz \ref{fig:fill_win} przedstawiają okno aplikacji, odpowiednio przed wypełnieniem danymi oraz po zakończeniu obliczeń.

\begin{figure}[h]
  \center
  \includegraphics[scale=0.4]{../image/empty_win.png}
  \caption{Początkowe okno aplikacji}
  \label{fig:empty_win}
\end{figure}

\begin{figure}[h]
  \center
  \includegraphics[scale=0.4]{../image/fill_win.png}
  \caption{Aplikacja po zakończonych obliczeniach}
  \label{fig:fill_win}
\end{figure}

Program posiada możliwość wczytania danych z pliku CSV, a następnie zapisanie danych wyjściowych również do pliku CSV lub TXT. Kolejnymi zaimplementowanymi funkcjonalnościami są:
\begin{enumerate}
  \item wyświetlenie danych wejściowych w oknie aplikacji
  \item wybór algorytmu rozkroju
  \item dodanie wielu długości podstawowych z różnym kosztem - domyślnie koszt jest równy długości.
  \item wyświetlenie długości podstawowych w oknie aplikacji
  \item dodanie szerkości cięcia dla metody brutalnej siły
  \item wyświetlenie wyniku w oknie aplikacji
\end{enumerate}

\subsection{Java}
Język programowania Java jest jezykiem obiektowym z elementami programowania funkcyjnego wprowadzonymi od wersji 8. Aplikacje stworzone w tej technologii mogą być stosowane w różnych systemach operacyjnych, gdyż programy napisane w języku Java są kompilowane do plików class które umieszczane są w skompresowanej paczce jar. Pliki class następnie są przetwarzane przez maszynę wirtualną Javy (JVM - Java Virtual Machine) do postaci bytecode który jest wykonywany na urządzeniu. Istnieją implementacje JVM na większość używanych platform.

Tworzenie aplikacji w technologii Java jest możliwe poprzez użycie zestawu JDK (Java Development Kit). Uruchamianie tych aplikacji jest możliwe w środowisku JRE (Java Runtime Environment)  (\hyperref[fig:java_arch]{rys.~\ref*{fig:java_arch}}).

\begin{figure}[h]
  \center
  \includegraphics[scale=0.4]{../image/java_arch.png}
  \caption{Elementy składowe technologi Java \cite{OracleJavaArch}}
  \label{fig:java_arch}
\end{figure}

Podstawowym elementem technologii jest maszyna wirtualna. Jest to element technologii odpowiedzialny za niezależność programów od specyfikacji urządzenia oraz systemu operacyjnego. JVM jest abstarakcyjną maszyną obliczeniową. Podobnie jak rzeczywiste urządzenia posiada zestaw instrukcji pozwlających na sterowanie nią oraz wyonywanymi zadaniami. Maszyna wirtualna Javy nie zna języka Java, jedynie jego postać binarną zapisana w plikach class. Pliki te zawierają instrukcje dla JVM lub bytecode oraz inne wymagane informacje. Wiele języków programowania wykorzystuje tą cechę maszyny wirtualnej. Wymaganiem jest aby program był w postaci poprawnego pliku class, aby mógł zostać wykonany na maszynie wirtualniej.

Technologia Java zawiera ponadto zestaw podstawowaych bibliotek pozwlających między innymi na budowanie plików JAR, refleksję czyli dostęp do metod oraz pól klasy bez zachowania zasad bezpieczeństwa, zdalne wywoływanie metod (RMI) oraz tworzenie graficznego interfejsu użytkownika Swing oraz AWT. Środowisko deweloperskie jest rozszerzone o narzędzia potrzebne do stworzenia programu, przykładowo: javac - kompilator przetwarzający pliki java do plików class, javadoc - narzędzie do tworzenia dokumentacji oraz język opisu interfejsów IDL służacy do komunikacji międzyprocesowej.
\subsection{Kotlin}
Kotlin jest obiektowym językiem programowania który jest interpretowany do bytecode wywoływanego na maszynie wirtualnej Javy. Kotlin w porównaniu z Javą wnosi usprawnienia do programownaia proceduralnego. Kotlin jest zgodny z językiem Java, odnosi to skutek w możliwości łączenia obu języków programowania. Jest to technologia podobna do języka Scala jednak czas kompilacji został skrócony. Jest to język silnie rozwijający się w środowisku programistycznym Androida. Dopiero najnowsza wersja narzędzi deweloperskich Androida pozwala na wykorzystywanie niektórych elementów Javy 8. Kotlin zmniejsza liczbę nadmiarowego kodu potrzebnego do napisania przez programistę. Głównymi celami stworzenia technologii Kotlin były: pełna kompatybilność z językiem Java, zwiększenia bezpieczeństwa względem Javy (null safe), bardziej elastyczny oraz nieskomplikowany kod. Jedną z najciekawszych funkcjonalności języka Kotlin jest tworzenie metod rozszerzających daną klasę. Przykładowo może zostać zdefiniowana metoda $isNotEmpty()$ dla klasy $String$:
\lstset{language=Java}
\begin{lstlisting}[frame=single]
  fun String.isNotEmpty() = !this.isEmpty()
\end{lstlisting}
Metoda ta będzię dostępna dla każdego obiektu typu $String$ w programie.
\subsection{JavaFX}

\section{Podsumowanie i wnioski}
Problem optymalnego rozkroju rur jest szczegółowym przypadkiem problemu pleckaowego. Szczególne rozwiązania tego problemu mogą zostać osiągnięte na wiele sposobów. Porównanie dwóch algorytmów obliczania optymalnego rozkroju rur - brutalnej siły oraz opóźnionej generacji kolumn wskazuje na dwa typy metod. Obejmują one metody służące do prototypowania oraz do dokładnego obliczania wartości rozkroju. Metoda "Brutal Force" jest mniej skomplikowana oraz jest szybsza niż metoda "Delayed Column Generation", dlatego może zostać wykorzystana do szybkiego protypowania oraz przewidywania szacunkowych kosztów rozkroju. Druga metoda może zostać zastosowana do dokładnego obliczenia schematów rozkrojów. Schematy te mogą zawierać więcej elementów niż zostało zamówione lecz nadal posiadać mniejszy koszt niż metoda brutalnej siły.

Obie porównywane metody posiadają wady i zalety. Porównując metodę BF do DCG można stwierdzić, że jest ona znacznie szybsza oraz daje bardziej homogeniczne rozkroje. Takie same długosci w jendym układzie powodują brak konieczności przestawiania noża podczas cięcia, skutkuje to mniejszym nakładem pracy podczas stosowania metody w warunkach rzeczywistych. Metoda ta odpowiednia jest do szybkiego prototypowania oraz szacowania kosztu. Przewagą metody DCG jest znaczna minimalizacja kosztu mimo większej liczby wykrojów i znacznie większego odpadu. Czas wykonania metody DCG powoduje iż, jest to metoda nieodpowiednia do planowania, lecz do określania konkretnych wykrojów. Jest to metoda bardziej oszczędna niż metoda BF.

Implementacja metody DCG użyta do testu, jest wariantem podstawowym zaprpopnowanym przez Gilmorea oraz Gomorego \cite{GilmoreGomoryV1Article}. Po wprowadzeniu usprawnień zaproponowanych w \cite{GilmoreGomoryV2Article} oraz optymalizacji implementacji, czas wykonania programu prawdopodobnie zostałby skrócony.

Aplikacja udostępnia możliwość obliczania rozkrojów z własnych zamówień, zadanym algorytmem. Porównanie implementacji obu metod rozkroju potwierdziło, iż metoda brutalnej siły jest znacznie szybsza lecz wyniki są gorsze niż metody opóźnionej generacji kolumn. Implementacja programu wymaga wielu obliczeń macierzowych oraz wielu obliczeń wartości maksymalnej z układu nierówności. Są to operacje o bardzo dużym zapotrzebowaniu czasowym. Aplikacja może zostać rozszerzona o obsługę przypadku gdy w trakcie metody opóźnionej generacji wystepuje ujemna wartość kolumny liczebności danego rozkroju. Obecnie gdy taka sytuacja wystąpi zwracany jest ostatni znany poprawny rozkrój. W trakcie przeprowadzania eksperymentu, wyniki te zostały odrzucone ze względu na możliwość, iż sytuacja ta spowodowana jest losowymi danymi, które mogły mieć nieprawidłowy format wejściowy, lub ze względu iż przypadek ten został pominięty w implementacji. Kolejnym usprawnieniem aplikacji może zostać podzielenie obliczeń na wątki, tak aby praca został zrównoleglona oraz przyspieszona.

Rozszerzeniem algorytmu zastosowanego w metodzie opóźninej generacji kolumn może zostać metoda medianowa, zaproponowana przez Gilmorea oraz Gomorego \cite{GilmoreGomoryV2Article}. Metoda ta według przeprowadzonych eksperymentów może skrócić czas oraz obiżyć zapotrzebowanie na zasoby obliczeniowe nawet o 90\%.


\listoffigures
\bibliographystyle{abbrv}
\bibliography{praca_magisterska}

\end{document}
