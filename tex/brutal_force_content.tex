\section{Metoda "Brutal Force"} \label{sec:bf}
\subsection{Algorytm wyjściowy}
Metoda ta opiera się zarówno na intuicji jak i na rozwiązaniu zaproponowanym przez Dantziga dla problemu plecakowego \cite{DantzigArticle}. Jest to metoda która w prosty sposób - nie używając złożonych modeli matematycznych, pozwala osiągnąć optymalny rozkrój materiału.

Pierwszym krokiem jest posortowanie elementów wyściowych malejąco wzgęldem ich długości $l_1 \ge l_2 \ge ... \ge l_m$ i umieszczenie w ten sposób w kolejce.

Drugim krokiem jest pobranie pierwszego elementu z kolejki i sprawdzenie, jak wiele razy jego długość zawiera się w długości elementu bazowego. Obliczone zostaje ile materiału pozostało w elemencie bazowym po docięciu najdłuższych elementów. Następnie pobierany jest kolejny odcinek z kolejki. Następuje sprawdzenie ile razy zawiera się on w pozostałej długości.
\begin{equation}\label{base_dantizg}
\begin{split}
& a_1 = [L/l_1],\\
& a_2 = [(L-l_1 a_1)/l_2],\\
& a_3 = [(L-(l_1 a_1+l_2 a_2))/l_3], ...
\end{split}
\end{equation}
Kroki te powtarzane są dopóki kolejka się nie skończy.

Każdy element wyjściowy posiada określoną liczebność jaką powinien osiągnąć pod koniec procesu cięcia. Jeśli na danym etapie procesu cięcia wymagana liczba elementów danego typu spada do zera, wówczas jest on pomijany w dalszej pracy algorytmu. Koniecznie jest sprawdzenie czy liczba uzyskanych elementów danego typu jest mniejsza lub równa od wymaganej:
\begin{itemize}
  \item Jeśli stwierdzenie jest prawdziwe - długość z której elementy są wycinane zostanie zmniejszona o liczbę wystąpień elementu pomnożoną przez jego długość, a licznik wymaganych odcinków danej długości zostanie zmniejszony o odpowiednią liczbę wystąpień
  \item Jeśli stwierdzenie jest fałszywe - długość z której elementy są wycinane zostanie zmniejszona o liczbę pozostałych wykrojów pomnożoną przez długość elementu, a licznik wymaganych odcinków danej długości zostanie ustawiony na zero.
\end{itemize}
Po zakończeniu przebiegu algorytmu dla jednego układu rozkroju, można określić ile razy będzie on użyty. Zostaje to wyznaczone poprzez obliczenie
\begin{equation}
g = floor\{min\{z_i/a_i\}\}, i \in [0..m], g \in Z
\end{equation}
gdzie $g$ to liczba ile razy dany schemat może zostać użyty, $z$ to liczbność wyjściowego elementu $i$ która pozostała do wycięcia, $a$ to ilość wykrojów elementu $i$ w bierzącym układzie, $m$ to liczba długości umieszczonych w rozkroju. Następnie licznik wymaganych odcinków elemntu $i$ zostaje zmniejszony o $g a_i$.

Cały proces powtarzany jest do momentu aż wszytskie wymagane elementy zostaną wycięte.

\subsection{Rozszerzenie o szerokość cięcia}
W warunkach rzeczywistych elementy wycinane są za pomocą ostrza które ma niezerową grubość. Wówczas metodę obliczania należy rozszerzyć jeśli ma odpowiadać warunkom rzeczywistym. Szerokość cięcia wlicza się w odpad. Jest kilka przypadków wliczania szerokości ostrza.

Jeżeli element jest równy długości bazowej wówczas nie wlicza się szerokości cięcia. Natomiast jeżeli materiał bazowy ma zostać pocięty na kilka elmentów wówczas do każdego dolicza się szerokość cięcia. Szczególnym przypadkiem jest, gdy ostatni element wraz z szerokością ostrza jest dłuższy niż długość odcinka, który został po wycięciu wcześniejszych elementów.

Gdyby szerokość cięcia nie zostałą uwzględniona w obliczeniach wówczas dla elementu wejściowego o długości 6000mm i wymaganych odcinkach 4500mm oraz 1500mm, obie długości zostały wycięte z jednego segmentu materiału bazowego. Skutkiem takiego postępowania byłby element krótszy o szerokość ostrza. Zazwyczaj długość ta może być akceptowana jako toleracncja dokładności maszyny. Jednak dla poprawności obliczeń wielkość ta powinna zostać uwzględniona.

\subsection{Rozszerzenie o wiele długości bazowych}
Dla zmniejszenia odpadu można użyć kilku długości bazowych. Rozszerzenie to wprowadza następująca zmianę algorytmu: obliczenia układu muszą zostać powtórzone dla każdego elementu wejściowego. Następnie wybierany jest ten rozkrój, który daje mniejszy odpad. Modyfikacja ta znacząco wpływa na wydajność metody. Jeżeli $n$ oznacza złożoność obliczeniową podstawowego algorytmu, a $m$ oznacza liczbę odcinków wejściowych, wówczas nowa złożonośc obliczeniowa wynosi $m*n$. % a może n^m

\subsection{Rozszerzenie o cenę materiału wsadowego}
Rozszerzenie to wprowadza zmianę koncepcyjną. Każdy element bazowy posiada cenę za metr bieżący materiału, umożliwia to obliczenie kosztu odpadu i wybranie tańszej opcji wykroju.

\clearpage

\subsection{Przykład}
\begin{enumerate}
  \item Dane wejściowe
  \begin{itemize}
    \item 6000mm - 3\$/mb
    \item 7000mm - 2\$/mb
    \item szerokość cięcia: 10mm
  \end{itemize}
  \item Dane wyjściowe
  \begin{itemize}
    \item 1x3500mm
    \item 1x3000mm
    \item 3x2000mm
    \item 5x500mm
  \end{itemize}
  \item Przebieg algorytmu
  \begin{itemize}
    \item Pierwszy rozkrój
    \begin{itemize}
      \item 3500mm mieści się raz w 6000mm. Zostaje $2500 - 10 = 2490$mm.
      \item 3000mm nie mieści się w 2490mm.
      \item 2000mm mieści się raz w 2490mm. Zostaje $490 - 10 = 480$mm.
      \item 500mm nie mieści się w 480mm.
      \item Rozkrój 6000mm: 3500mm, 2000mm. Odpad $6000 - 5500 = 500 * 0.003 = 1.5\$$
      \item ------------
      \item 3500mm mieści się dwa razy w 7000mm. Dostępny jest jeden odcinek 3500mm. Zostaje $3500 - 10 = 3490$mm.
      \item 3000mm mieści sie raz w 3490mm. Zostaje $490 - 10 = 480$mm.
      \item 2000mm nie mieści się w 480mm.
      \item 500mm nie mieści się w 480mm.
      \item Rozkrój 7000mm: 3500mm, 3000mm. Odpad $7000 - 6500 = 500 * 0.002 = 1.0\$$
      \item ------------
      \item Wybrano rozkrój 3500mm, 2000mm na długości 7000mm ze względu na mniejszy koszt odpadu.
      \item Do realizacji posostało: 0x3500mm; 0x3000mm; 3x2000mm; 5x500mm
    \end{itemize}
    \item Drugi rozkrój
    \begin{itemize}
      \item 2000mm mieści się trzy razy w 6000mm. Uwzględniając szerokość cięcia - zostaną użyte tylko dwa elementy od długości 2000mm. Zostaje $2000 - 2*10 = 1980$mm.
      \item 500mm mieści się trzy razy w 1980mm. Zostaje $480 - 3*10 = 450$mm.
      \item Rozkrój 6000mm: 2x2000mm, 3x500mm. Odpad $6000 - 5500 = 500 *0.003 = 1.5\$$
      \item ------------
      \item 2000mm mieści się trzy razy w 7000mm. Zostaje $1000 - 3*10 = 970$mm.
      \item 500mm mieści się raz w 970mm. Zostaje $470 - 10 = 460$mm.
      \item Rozkój 7000mm: 3x2000mm, 500mm. Odpad $7000 - 6500 = 500 * 0.002 = 1.0\$$
      \item ------------
      \item Wybrano rozkrój 3x2000mm, 500mm na długości 7000mm ze względu na mniejszy koszt odpadu
      \item Do realizacji posostało: 0x3500mm, 0x3000mm, 0x2000mm, 4x500mm
    \end{itemize}
    \item Trzeci rozkrój
    \begin{itemize}
      \item 500mm mieści się dwanaście razy w 6000mm. Dostępne są cztery element 500mm. Zostaje $6000 - 4*500 - 4*10 = 3960$mm.
      \item Rozkrój 6000mm: 4x500mm. Odpad $6000 - 4*500 = 4000 * 0.003 = 12\$$
      \item ------------
      \item 500mm mieści się czternaście razy w 7000mm. Dostępne są cztery elementy 500mm. zostaje $7000 - 4*500 - 4*10 = 4960$mm
      \item Rozkrój 7000mm: 4x500mm. Odpad $7000 - 4*500 = 5000 * 0.002 = 10\$$
      \item ------------
      \item Wybrano rozkrój 4x500 na długości 7000mm ze względu na mniejszy koszt odpadu
      \item Do realizacji posostało: 0x3500mm, 0x3000mm, 0x2000mm, 0x500mm
    \end{itemize}
    \item Podsumowanie
    \begin{itemize}
      \item Rozkroje : 3500mm, 2000mm na długości 7000mm; 3x2000mm, 500mm na długości 7000mm; 4x500 na długości 7000mm.
      \item Suma odpadów: $6000 * 0.002 = 12\$$
    \end{itemize}
  \end{itemize}
\end{enumerate}

\clearpage

\subsection{Podsumowanie}
Przedstawiony algorytm jest intuicyjny oraz zwraca poprawne wyniki. Główną wadą jest brak świadomości o następnym kroku oraz kolejnych wykrojach. Dla przykładu: Zosatło 1000mm materiału, do dyspozycji (z długości mniejszych niż 1000mm) jest odcinek 900mm oraz dwa elementy 480mm. Algorytm przydzieli odcinek 900mm, jednak lepszym wyborem byłoby użycie dwóch odcinków 480mm.
