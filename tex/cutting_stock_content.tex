\section{Cutting Stock Problem - Problem optymalnego rozkroju}
\label{sec:cuttingStockProblem}
\newcommand{\mf}[1]{\textbf{\textit{#1}}}
\newcommand{\tsub}[1]{\textsubscript{#1}}
\newcommand{\tsuper}[1]{\textsuperscript{#1}}

Problem optymalnego rozkroju jest problemem wykroju zadanej liczby elementów z wielu elementów podstawowych takich, jak rury, arkusze papieru lub metalu, w taki sposób aby zminimalizować niewykorzystany materiał (odpad). Jest to problem optymalizacyjny znajdujący zastosowanie głównie w przemyśle.  W odniesieniu do  złożoności obliczeniowej jets to problem z rodziny problemów $\mathcal{NP}$-Trudnych, który może zostać zredukowany do problemu plecakowego (\hyperref[sec:knapsackProblem]{rozdział~\ref*{sec:knapsackProblem}}). W rozdziale tym zostanie opisany jednowymiarowy problem optymalnego rozkoroju.

\subsection{Metoda "Delayed Column Generation"}
Metoda ta została zparoponowana przez Gilmore'a i Gomorego w 1961 roku \cite{GilmoreGomoryV1Article}. Gdy problem optymalnego rokroju zostanie sformułowany jako problem programowania całkowitego wówczas liczba zmiennych wchodzących w skład równań powoduje że rozwiązanie jest nieosiąglane. Dla przykładu gdy podstawowa długość to 200 z której ma zostać wycięte 40 różnych elementów o długościach od 20 do 80 wówczas liczba róznych wzorców rozkroju może osiągnąć nawet 100 milionów. Czas potrzebny do przejścia po samych rozkrojach byłby niosiagalny. Metoda ta pozwala na ciągłą generację nowych rozwiązań. Jest ona również metodą która znosi restrykcję liczb całkowitych w trakcie obliczania wyniku, dlatego wynik zostaje zaokraglony w górę, co odnosi skutek w tym że jest produkowane więcej lub tyle samo elementów niż jest wymagane przez zlecenie. Wynikiem tej metody jest rozwiązanie najbliższe optymalnemu.
\subsubsection{Wprowadzenie}
Założeniem metody jest że zamówienie $N_i$ elementów długości $l_i$, gdzie $i=1,2,\dots,m$, wyciętych z rur długości początkowych $L_1, L_2, \dots, L_k$, dla którego spełniony jest warunek, że isntiej takie $j$ że dla każdego $i$ spełnina jest nierówność $L_j \ge l_i$. Całkowity koszt rozkrojów jest całkowitym kosztem użytych elementów podstawowych. Celem rozwiązania problemu jest otrzymanie tylu wykrojów ile jest wymaganych przez zamówienie przy jak najmniejszym koszcie. Warunkiem koniecznym aby zamówinie zostało spełniona jets nierówność
\begin{equation*}
a_{i1}x_1 + a_{i2}x_2 + \dots + a_{in}x_i \ge N_i, \qquad i = 1, \dots, m
\end{equation*}
gdzie $a_{ij}$ oznacza ile razy w danym schmemacie rozkroju $x_i$ została użyta długość $l_i$. Funkcją kosztu która powinna zostać zminimalizowana wynosi
\begin{equation}\label{eq:base_cost}
  c_1x_1 + c_2x_2+\dots+c_nx_n
\end{equation}
gdzie  $c_i$ to koszt długości podstawowej z której jest pobierany $i$-ty wykrój. Wprowadzenie dodatkowych zmiennych $x_{n+1},\dots,x_{n+m}$ pozwalają opisać problem optymalnego rozkroju jako problem znalezienia takich liczb całkowitych $x_1,\dots,x_{n+m}$ spełniających
\begin{align}
    a_{i1}x_1+\dots+a_{in}x_n-x_{n+i} & =N_i, & \qquad i=1,\dots,m  \label{eq:additional_base}\\
    x_j & \ge 0, & \qquad j=1,\dots,n+m \label{eq:additional_base_condition}
\end{align}
oraz dla których (\ref{eq:base_cost}) jest jak najmniejsze.

Takie sformułowanie problemu jest niepraktyczne ze względu na ograniczenie do liczb całkowitych oraz z powodu że $n$ może być bardzo duże nawet gdy ilość $k$ elementów podstawowych, jak i ilość $m$ zamówionych długości jest umiarkowana.

Jeśli zostanie usunięty warunek całkowitości rozwiązania wówczas rozwiązanie będzie należało do zbioru liczb rzeczywistych dodatnich. Rozwiązanie to może zostać zaokrąglone w górę lecz wtedy może zostać wyprodukowane więcej elementów niż zostało zamówione. Rozwiązanie może być również zaookrąglana na przemina w górę i w dół, a elementy które nie spełniają założeń zamówienia są dodawane do wykrojów metodą \textit{ad hoc}. Gdy wartości niecałkwite są duże wówczas zaokrąglenie jej nie wpływa znacząco na koszt, jednak gdy wartości są rzędu dziesiątek wówczas zaokrąglenie ma znaczny wpływ na koszt. Omawiana metoda znosi ograniczenie dla liczb całkowitych.

Usunięty warunek całkowitości odnosi skutek w tym, że zmienne dodtkowe mogą zostać usunięte z równiania (\ref{eq:additional_base}). Dopóki rozwiązania (\ref{eq:additional_base}) oraz (\ref{eq:additional_base_condition}) zawierają dodatnie zmienne dodatkowe wówczas istnieje rozwiązanie o takim samym koszcie w którym nie zawierają się dodatnie zmienne dodatkowe. Niech $\bar{x}_1, \dots, \bar{x}_n, \bar{x}_{n+1}, \dots, \bar{x}_{n+m}$ będzie rozwiązaniem (\ref{eq:additional_base}) oraz (\ref{eq:additional_base_condition}) dla którego $\bar{x}_{n+1} \neq 0$. Dla tego rozwiązania istnieje takie $i$ dla którego $a_{1i}\bar{x}_i \ge \bar{x}_{n+1}$, to jest, $i$-ty schemat rozkroju należy do rozwiązania w przynajmniej takiej liczebności aby zamówinie długości $l_1$ było spełnione. Jeśli nie istnieje takie $i$ które spełnia warunek wówczas, niech $j$-ta zmienna przyjmuje niezerową wartość $\bar{x}_j$ oraz niech $k$-ty rozkrój będzie identyczny jak $j$-ty z wyłączeniem uwzględniania długości $l_1$. W takim przypadku w $k$-tym rozkroju długość $l_1$ która została uwzględniona w $j$-tym rozkroju trkatowana jest jako odpad. Rozwiązanie  $\bar{x}_1{'}, \dots, \bar{x}_n{'}, \bar{x}_{n+1}{'}, \dots, \bar{x}_{n+m}{'}$ z tym samym kosztem co poprzednio zostało uzyskane poprzez przypisanie $\bar{x}_i{'} = \bar{x}_i$ dla $i \neq j$, $k$, $n+1$: $\bar{x}_j{'}=0, \bar{x}_k{'}=\bar{x}_k+\bar{x}_j$ oraz $\bar{x}_{n+1}{'}=\bar{x}_{n+1}- a_{1j}\bar{x}_j$ ponieważ koszt zmiennych $x_j$ oraz $x_k$ jest taki sam. W nowym rozwiązaniu zmienna $x_{n+1}$ została zredukowana. Jeśli nie została zmniejszona o tyle aby $a_{1i}\bar{x}_i{'} \ge \bar{x}_{n+1}{'}$ wtedy powyższy proces jet powtarzany dopóki nie zostanie znalezione rozwiązanie w którym jedna zmienna nie spełnia nierówności. Jeśli a_{1i}\bar{x}_i{'} \ge \bar{x}_{n+1}{'}$ jest spełnione wówczas zmienna dodatkowa $x_{n+1}$ może być traktowana jako zmienna z przechowywaną wartością $0$ w rozwiązaniu z takim smaym kosztem jak powyższe rozwiązanie. Niech $k$-ty rozkrój będzie schematem identyczny jak $j$-ty rozkrój z wyłączeniem długości $l_1$ oraz niech okreslna nowe rozwiązanie  $\bar{x}_1{'}, \dots, \bar{x}_n{'}, \bar{x}_{n+1}{'}, \dots, \bar{x}_{n+m}{'}$ poprzez przypisanie $\bar{x}_i{'}=\bar{x}_i$ dla $i \neq j, k, n+1$, $\bar{x}_j{'} = \bar{x}_j - (\bar{x}_{n+1})/a_{1j},\bar{x}_k{'}=\bar{x}_k+(\bar{x}_{n+1})/a_{1j}$ oraz $\bar{x}_{n+1}{'} = 0$. Ponieważ współczynniki odpowiedzialne za koszt są identyczne dla $x_j$ oraz $x_k$, dlatego nowe rozwiązanie posiada taki sam koszt jak poprzednie rozwiązanie.

Zniesienie warunku cąłkowitości rozwiązania pozawala pominąć zmienne dodatkowe, jednak w pewnych przypadkach jest zalecane pozostawienie ich. Bez zmiennych dodatkowych każde minimalne rozwiązanie zawiera zazwyczaj $m$ schematów rozkroju, podczas gdy rozwiązanie ze zmiennymi dodatkowymi może zawierać mniej niż $m$ rozkrojów. Opisywana metoda nie znosi zmiennych dodatkowych.

Metoda simplex która jest stosowana do obliczenia dopuszczalnego rozwiązania (\ref{eq:additional_base}) w odniesieniu do (\ref{eq:additional_base_condition}) dla którego (\ref{eq:base_cost}) jest namjniejsze. Dla podstawowego rozwiązania (\ref{eq:additional_base_condition}) oraz (\ref{eq:base_cost}), metodą simplex sprawdzane są inne zmienne które mogą zastąpić pewne zmienne w bierzącym rozwiązaniu. Niech bierzącym rozwiązniem będzie $x_1,x_2,\dots,x_m$. Niech \mf{P}\tsub{i} będzie wektorem $[a_{1i}, a_{2i}, \dots, a_{mi}]$ oraz niech $c_i$ bedzie kosztem w (\ref{eq:base_cost}) który jest powiązany ze zmienną $x_i$. Jeśli $x_i$ jest zmienną dodatkową wówczas koszt wynosi $0$, a wektor ma jedną niezerową współrzędną wynoszącą $-1$. Niech $\textbf{\textit{P}} = [a_1,a_2,\dots,a_m]$ określa nowy schemat rozkroju który używa długości bazowej $L$ o koszcie $c$. Następnie niech \mf{A} będzie macierzą której kolumnami są wektory $\mf{P}_1,\dots,\mf{P}_m$. Ponieważ $\mf{P}_1, \dots, \mf{P}_m$ określą podstawę macierzy, wektor kolumnowy \mf{U} spełnia równość
\begin{equation}\label{eq:uMatrixDef}
  \mf{A} \cdotp \mf{U} = \mf{P}.
\end{equation}
Nowy schemat rozkroju może zostać uzyty w rozwiązaniu jako jego ulepszenie wtedy i tylko wtedy, gdy
\begin{equation}\label{eq:costMatrixEq}
  \mf{C} \cdotp \mf{U} > c
\end{equation}
gdzie \mf{C} jest wektorem wierszowym ze współczynnikami $c_1,c_2,\dots,c_m$. Jeśli wektor wierszowy $\mf{C} \cdotp \mf{A}\tsuper{-1}$ posiada współczynniki $b_1,\dots,b_2$, wtedy z równań (\ref{eq:uMatrixDef}) oraz (\ref{eq:costMatrixEq}) można wywnioskować że istnieje taki rozkrój z elementu podstawowego o długości $L$, który może poprawić rozwiązanie wtety i tylko wtedy, gdy istnieją nieujemne liczby całkowite $a_1,\dots,a_m$ spełniające nierówności
\begin{align}
L \ge l_1a_1+\dots+l_ma_m \label{length_eq} \\
b_1a_1+\dots+b_ma_m > c. \label{cost_eq}
\end{align}
$\mf{C} \cdotp \mf{A}\tsuper{-1}$ zawsze jest częścią rozwiązania normalnej metody simplex.

Jeśli istnieje taka nieujemna liczba całkowita $a_i$ która spełnia nierówności (\ref{length_eq}) oraz (\ref{cost_eq}), wówczas isnieje taka nieujemna liczba całkowita która jest rozwiązaniem nierówności (\ref{length_eq}) dla której $b_1a_1+\dots+b_ma_m$ jest maksymalne. Problem wyboru nowej zmiennej dla metody simplex może zostać wyrażony poprzez rozwiązanie $k$ problemów pomocniczych (po jednym dla każdej długości bazowej), które są całkowitymi problemami programowania liniowego. Problemy te mogą zostać rozwiązane poprzez programowanie dynamiczne lub metodą \textit{ad hoc}.

Jako, że maksymalizacja $b_1a_1+\dots+b_ma_m$ w odniesieniu od (\ref{length_eq}) jest generalizacją problemu plecakowego, dlatego można rozwiązać go metoda opisaną przez Dantziga \cite{DantzigArticle} oraz w (\hyperref[knapsack:dynamicProgramming]{sekcji~\ref*{knapsack:dynamicProgramming}}). Niech $F_s(x)$ będzie wartością maksymalną $b_1a_1+\dots+b_sa_s$ w odniesieniu do nierówności $x \ge l_1a_1+\dots+l_sa_s$, wówczas
\begin{equation*}
  F_{s+1}(x) = max_r\{rb_{s+1}+F_s(x-rl_{s+1})\},
\end{equation*}
gdzie $r$ może zostać wybrane z zakresu $0 \le r \le [x/l{s+1}]$, a nawiasy kwadratowe oznaczają największą częśc całkowitą z wyrażenia. Tylko jedno kompletne obliczenie wyrażenia programowania dynamicznego jest niezbędne aby wprowadzić nową zmienną do metody simplex. Gdy najdłużsyzm elementem jest $L_1$, wówczas automatycznie zostaną obliczone pozostałe długości.

Programowanie dynamiczne często wykonuje więcej obliczeń niż jest konieczne. Aby przyspieszyć proces możliwe jest użycie metody podobnej do zapropnowanej przez Dantziga \cite{DantzigArticle} oraz opisanej w (\hyperref[brutalForce]{sekcji~\ref*{brutalForce}}). Niech $i_1,\dots,i_m$ będą takie, że $b_i_1/l_i_1 \ge b_i_1/l_i_2 \ge \dots \ge b_i_m/l_i_m$. Następnie obliczone zostają współczynniki $a_i_1=[L/l_i_1], a_i_2=[(L-l_i_1a_i_1)/l_i_2], a_i_3 = [(L - (l_i_1a_i_1 + l_i_2a_i_2))/l_i_3], etc$. Dopiero gdy proste metody nie dostarczą rozwiązania, powinny zostać użyte bardziej złożone metody, jak programowanie dynamiczne.

\subsubsection{Algorytm}

\begin{enumerate}
\item Określnie $m$ początkowych rokrojów i ich kosztu w następujący sposób: dla każdego $i$ wybranie długości bazowej $L_j$ dla której $L_j > l_i$ i określenie $i$-tego rokroju jako wycięcia $a_{ii} = [L_j / l_i]$ elementów o długości $l_i$ z $L_j$. Koszt $i$-tego rozkroju będzie równy cenie $c_j$ długości $L_j$ z której $i$-ta operacja wycina odcinki o długości $l_i$.
\item Uformowanie macierzy \mf{B}
\[
\begin{matrix}
1 & -c_1  & -c_2  & \dots & -c_m \\
0 & a_{11}  & 0 & \dots & 0 \\
0 & 0 & a_{22}  & \dots & 0 \\
\vdots&\vdots&\vdots&\ddots&\vdots \\
0 & 0 & 0 & \dots & a_{mm}
\end{matrix}
\]
gdzie $a_{ii}$ jest ilością odcinków o długości $l_i$ wyciętych w $i$-tym rozkroju z długości bazowej o koszcie $c_j$. Ostatnie $m$ kolumn jest odpowiada kolejmnym rozkrojom. Dane te będą aktualizowane gdy zostanie znaleziony wynik który zmniejsyz koszt rozwiązania.

Utworzenie $m$ $m+1$ wymiarowych wektorów kolumnowych \mf{S\tsub{1}},...,\mf{S\tsub{m}} odnoszących się do zmiennych dodatkowych, gdzie \mf{S\tsub{i}} zawiera same zera z wyjątkiem wiersza $(i+1)$ któryc przechowuje wartość $-1$. Stworzony również zostaje $m+1$ wymiarowy wektor kolumnowy \mf{N'} który jako pierwszy element przyjmuje 0, a w następnych $i$-tych wierszach posiada wartości $N_i$.

Obliczenie macierzy \mf{B}\tsuper{-1} która wynosi:
\[
\begin{matrix}
1 & c_1/a_{11}  & c_2/a_{22}  & \dots & c_m/a_{mm} \\
0 & 1/a_{11}  & 0 & \dots & 0 \\
0 & 0 & 1/a_{22}  & \dots & 0 \\
\vdots&\vdots&\vdots&\ddots&\vdots \\
0 & 0 & 0 & \dots & 1/a_{mm}
\end{matrix}
\]

Niech $\mf{N} = \mf{B}^{-1} \cdotp \mf{N'}$. Sprawdzając czy pierwszy element z $\mf{B}\tsuper{-1} \cdotp \mf{P}$ jest dodatni można określić czy istnieje możliwość polepszenia rozwiązania. Wektor kolumnowy \mf{P} jest wektorem złożonym ze zmiennych nieużytych w bieżącym rozwiązaniu. Dla przykładu pierwszy element jest kosztem pomnożonym przez $-1$, a pozostałe $m$ wierszy jest równe zmiennym $a_{ij}$.

\item Z powyższego puntku wynika że jeśli $i$-ta zmienna dodatkowa która nie wchodzi w skład rozwiązania, może ulepszyć rozwiązanie wtedy i tylko wtedy, gdy $(i+1)$ element pierwszego wiersza \mf{B}\tsuper{-1} jest ujemny.

\item Jeśli nie jest możliwe zmniejszenie kosztu rozwiązania należy określić czy wprowadznie nowego rozkroju poprawi rozwiązanie. Jest to możliwe poprzez sprawdznie czy dla $L$ z kosztem $c$ istnieje rozwiązanie nierówności (\ref{length_eq}) oraz \ref{cost_eq}, gdzie $b_1,\dots,b_m$ to ostatnie $m$ elementów z pierwszego wiersza \mf{B}\tsuper{-1}. Jeśli te nierówności nie posiadają rozwiązania dla dowolnej długości $L_1,\dots,L_k$ z kosztem odpowiednio $c_1,\dots,c_m$, wówczas bieżące rozwiązanie jest optymalne. Rozwiązanie i jego koszt jest określone poprzez \mf{N}, gdzie pierwszy wiersz odpowiada cenie, a pozostałe $m$ wierszy jest, w kolejności, odpowiednimi wartościami $m$-tej kolumny z \mf{B}\tsuper{-1}.

Jeśli nowy rozkrój zmniejsza koszt rozwiązania, zostaje uformawany nowy wektor \mf{P} o współczynnikamch, w kolejności $-c,a_1,a_2,\dots,a_m$.

\item Wprowadznie zarówno dodatkowej zmiennej jak i nowego rozkroju może poprawić rozwiązanie. W obu przypadkach \mf{P} będzie wektorem kolumnowym. Określenia nowych \mf{B}\tsuper{-1} oraz \mf{N} które opisują ulepszone rozwiązanie i jego koszt, zostaje osiągnięte poprzez przejście kroków 3, 4 oraz kontynujacje kroku 5 w nastepujący sposób: Obliczenie $\mf{B}\tsuper{-1} \cdotp \mf{P}$ - niech wynikiem będą elementy $y_1,\dots,y_m,y_{m+1}$ oraz niech elementami bierzącego wektora \mf{N} będą $x_1,\dots,x_m,x_{m+1}$. Ustalenie $i$, $ i \ge 2$ dla którego $y_i > 0$, $x_i \ge 0$ oraz $x_i/y_i$ jest najminiejsze, a następnie przypisanie tej wartości do zmiennej $k$.

Jeśli stosunek nie jest równy zeru, wówczas $k$-ty element wektora \mf{P}, $y_k$, będzie elementem wokół którego zajdzie eliminacja Gaussa, odbywająca się równocześnie w \mf{B}\tsuper{-1}, $\mf{B}\tsuper{-1} \cdotp \mf{P}$ oraz \mf{N}. Eliminacja ta przebiega dla macierzy $(m+1) \times (m+3)$ wymiarowej \mf{G} uformowanej z \mf{B}\tsuper{-1} poprzez dołączenie kolumn $\mf{B}\tsuper{-1} \cdotp \mf{P}$ oraz \mf{N}. Pierwsze $m+1$ kolumn \mf{G'} formuje nową macierz \mf{B}\tsuper{-1}, a kolumna $m+2$ jest nowym wektorem \mf{N}. Zależność między kolumnami \mf{B}\tsuper{-1}, a rozkrojami lub zmiennymi dodatkowymi jest aktualizowana poprzez usunięcie $k$-tej kolumny i podmienieniu jej na nowy rozkrój lub zmienną dodatkową.

\end{enumerate}

\subsubsection{Metody użyte w implementacji}

Do obliczenia wartości maksymalnej nierówności która zostanie przypisana do \mf{P} zostały użyte metody:
\begin{enumerate}
  \item Dwufazowa metoda simplex - metoda to znajduje zastosowanie gdy bierzące rozwiązanie układu jest ujemne. Zwykła metoda sympleks jest użyta w drugiej fazie omawianej procedury. Faza pierwsza polega na przeprowadzeniu obliczeń metodą simplex ze zmienioną funcją celu. Jeśli zmienna wchodząca w skład rozwiązania układu jest ujemna wówczas do danego równania dodawana jest dodatkowa sztuczna zmienna. Funkcja celu wówczas przyjmuje postać sumy zmiennych które zostały dodane jako sztuczne do równań o ujemnym rozwiązaniu. Po obliczeniu wartości fazy pierwszej, nastepuje ponowne przekształcenie funkcji celu i przeprowadzenie normlanej procedury sympleks, jako fazy 2.
  \item Metoda podziału i ograniczeń - metoda ta pozwala osiągnąć wyniki całkowite z rozwiązań układów nierówności. Polega ona na budowie drzewa binarnego. Każdy liść staje się rodzicem poprzez stworznie dwóch węzłów poprzez sprawdzenie dwóch warunków. Lewy potomek tworzony jest z dodatkowym warunkiem $x_i \le \lfloor c_i \rfloor$ gdzie $c_i$ jest zmienną niecałkowitą wchodzącą w skład rozwiązania. Prawy potomek posiada warunek $x_i \ge \lceil c_i \rceil$. Następnie dla każdego węzła przeprowadzana jest metoda sympleks. Jeśli dany węzeł posiada rozwiązanie wówczas procedura ta jest powtarzana, aż do osiągnięcia wyniku całkowitego. Poszczególne warunki dołączane są do układu nierówności który przekazywany jest do kolejnych potomków. Jeśli tworzenie drzewa binarnego jest zakończone, wówczas jako rozwiązanie wybierany jest liść z jak największą wartością zwróconą przez metodę simplex.
\end{enumerate}

\subsubsection{Przykład}

Zamówione zostało 20 elementów o długości 2, 10 o długości 3 oraz 20 o długości 4. Jako długości bazowe zostały określone elementy o dłogości 5 z ceną 6, 6 z ceną 7 oraz o długości 9 z ceną 10. \newline
Początkowo:
\begin{equation*}
  \begin{aligned}
    &\mf{B} =
    \begin{bmatrix}
      1.0 & -6.0 & -6.0 & -6.0 \\
      0.0 & 2.0 & 0.0 & 0.0 \\
      0.0 & 0.0 & 1.0 & 0.0 \\
      0.0 & 0.0 & 0.0 & 1.0
    \end{bmatrix}, \qquad
    &\mf{N'} =
    \begin{bmatrix}
      0.0\\
      20.0\\
      10.0\\
      20.0
    \end{bmatrix} \\
    &\mf{B}^{-1} =
    \begin{bmatrix}
      1.0 & 3.0 & 6.0 & 6.0 \\
      0.0 & 0.5 & 0.0 & 0.0 \\
      0.0 & 0.0 & 1.0 & 0.0 \\
      0.0 & 0.0 & 0.0 & 1.0
    \end{bmatrix}, \quad
    &\mf{N} =
    \begin{bmatrix}
      240.0\\
      10.0\\
      10.0\\
      20.0
    \end{bmatrix}
  \end{aligned}
\end{equation*}
Długości bazowe będą próbowane w kolejności malejącej ponieważ im dłuższy element, tym więcej możliwości rozkroju. Pierwszy układ nierówności:
\begin{equation*}
  \begin{aligned}
    2.0 x_{1}+ 3.0 x_{2}+ 4.0 x_{3} & &\le 9.0 \\
    3.0 x_{1}+ 6.0 x_{2}+ 6.0 x_{3} & &> 10.0
  \end{aligned}
\end{equation*}
Rozwiązaniem nierówności jest $(0.0,3.0,0.0)$. Wówczas wektor $\mf{P} = [-10.0, 0.0, 3.0, 0.0]$ oraz
\begin{equation*}
  \begin{aligned}
    \mf{G\tsuper{'}} =
    \begin{bmatrix}
      1.0 & 3.0 & 6.0 & 6.0 & 240.0 & 8.0  \\
      0.0 & 0.5 & 0.0 & 0.0 & 10.0 & 0.0 \\
      0.0 & 0.0 & 1.0 & 0.0 & 10.0 & 3.0 \\
      0.0 & 0.0 & 0.0 & 1.0 & 20.0 & 0.0
    \end{bmatrix}
  \end{aligned}
\end{equation*}
gdzie ostatnią kolumną jest $\mf{B}^{-1} \cdotp \mf{P}$. Jako element psiowy wokół ktorego zajdzie eliminacja Gaussa to wrtość z ostatniej kolumny $3.0$.
Macierzą po eliminacji Gaussa jest:
\begin{equation*}
  \mf{G\tsuper{'}} =
  \begin{bmatrix}
    1.0 & 3.0 & 3.33 & 6.0 & 213.33 & 0.0 \\
    0.0 & 0.5 & 0.0 & 0.0 & 10.0 & 0.0 \\
    0.0 & 0.0 & 0.33 & 0.0 & 3.33 & 1.0 \\
    0.0 & 0.0 & 0.0 & 1.0 & 20.0 & 0.0
  \end{bmatrix}
\end{equation*}
Zmieniona zostaje druga nierówność na $3.0 x_{1}+ 3.33 x_{2}+ 6.0 x_{3} > 10.0$. Wektor \mf{P} dla takiego układu wynosi $[-10.0,3.0,1.0,0.0]$ oraz
\begin{equation*}
  \mf{G\tsuper{'}} =
  \begin{bmatrix}
    1.0 & 3.0 & 3.33 & 6.0 & 213.33 & 2.33 \\
    0.0 & 0.5 & 0.0 & 0.0 & 10.0 & 1.5 \\
    0.0 & 0.0 & 0.33 & 0.0 & 3.33 & 0.33 \\
    0.0 & 0.0 & 0.0 & 1.0 & 20.0 & 0.0
  \end{bmatrix}
\end{equation*}
gdzie element osiowy wynosi $1.5$. Po eliminacji Gaussa:
\begin{equation*}
  \mf{G\tsuper{'}} =
  \begin{bmatrix}
    1.0 & 2.22 & 3.33 & 6.0 & 197.78 & 0.0 \\
    0.0 & 0.33 & 0.0 & 0.0 & 6.67 & 1.0 \\
    0.0 & 0.11 & 0.33 & 0.0 & 1.11 & 0.0 \\
    0.0 & 0.0 & 0.0 & 1.0 & 20.0 & 0.0
  \end{bmatrix}
\end{equation*}
Zmodyfikowana nierówność wynosi $ 2.22 x_{1}+ 3.33 x_{2}+ 6.0 x_{3} > 10.0$. Wektor \mf{P} dla takiego układu wynosi $[-10.0,0.0,0.0,2.0]$ oraz
\begin{equation*}
  \mf{G\tsumper{'}} =
  \begin{bmatrix}
    1.0 & 2.22 & 3.33 & 6.0 & 197.78 & 2.0 \\
    0.0 & 0.33 & 0.0 & 0.0 & 6.67 & 0.0 \\
    0.0 & 0.11 & 0.33 & 0.0 & 1.11 & 0.0 \\
    0.0 & 0.0 & 0.0 & 1.0 & 20.0 & 2.0
  \end{bmatrix}
\end{equation*}
gdzie element osiowy wynosi $2.0$. Po eliminacji Gaussa:
\begin{equation*}
  \mf{G\tsuper{'}} =
  \begin{bmatrix}
    1.0 & 2.22 & 3.33 & 5.0 & 177.78 & 0.0 \\
    0.0 & 0.33 & 0.0 & 0.0 & 6.67 & 0.0 \\
    0.0 & 0.11 & 0.33 & 0.0 & 1.11 & 0.0 \\
    0.0 & 0.0 & 0.0 & 0.5 & 10.0 & 1.0
  \end{bmatrix}
\end{equation*}
Zmodyfikowana nierówność wynosi $ 2.22 x_{1}+ 3.33 x_{2}+ 5.0 x_{3} > 10.0$. Wektor \mf{P} dla takiego układu wynosi $[-10.0,1.0,1.0,1.0]$ oraz
\begin{equation*}
  \mf{G\tsumper{'}} =
  \begin{bmatrix}
    1.0 & 2.22 & 3.33 & 5.0 & 177.78 & 0.56 \\
    0.0 & 0.33 & 0.0 & 0.0 & 6.67 & 0.33 \\
    0.0 & 0.11 & 0.33 & 0.0 & 1.11 & 0.22 \\
    0.0 & 0.0 & 0.0 & 0.5 & 10.0 & 0.5
  \end{bmatrix}
\end{equation*}
gdzie element osiowy wynosi $0.5$. Po eliminacji Gaussa:
\begin{equation*}
  \mf{G\tsuper{'}} =
  \begin{bmatrix}
    1.0 & 2.5 & 2.5 & 5.0 & 175.0 & 0.0 \\
    0.0 & 0.5 & 0.5 & 0.0 & 5.0 & 0.0 \\
    0.0 & 0.5 & 1.5 & 0.0 & 5.0 & 1.0 \\
    0.0 & 0.25 & 0.75 & 0.5 & 7.5 & 0.0
  \end{bmatrix}
\end{equation*}
Układ nierówności:
\begin{equation*}
  \begin{aligned}
    2.0 x_{1}+ 3.0 x_{2}+ 4.0 x_{3} &\le 9.0 \\
    2.5 x_{1}+ 2.5 x_{2}+ 5.0 x_{3} &> 10.0
  \end{aligned}
\end{equation*}
nie posiada rozwiązania całkowitego. Wówczas brana jest następna długośc podstawowa 6. Nowy układ wynosi:
\begin{equation*}
  \begin{aligned}
    2.0 x_{1}+ 3.0 x_{2}+ 4.0 x_{3} &\le 6.0 \\
    2.5 x_{1}+ 2.5 x_{2}+ 5.0 x_{3} &> 7.0
  \end{aligned}
\end{equation*}
dla którego wektor \mf{P} wynosi $[-7.0,1.0,0.0,1.0]$ oraz
\begin{equation*}
  \mf{G\tsumper{'}} =
  \begin{bmatrix}
    1.0 & 2.5 & 2.5 & 5.0 & 175.0 & 0.5 \\
    0.0 & 0.5 & 0.5 & 0.0 & 5.0 & 0.5 \\
    0.0 & 0.5 & 1.5 & 0.0 & 5.0 & 0.5 \\
    0.0 & 0.25 & 0.75 & 0.5 & 7.5 & 0.75
  \end{bmatrix}
\end{equation*}
gdzie element osiowy wynosi $0.75$. Po eliminacji Gaussa:
\begin{equation*}
  \mf{G\tsuper{'}} =
  \begin{bmatrix}
    1.0 & 2.33 & 3.0 & 4.67 & 170.0 & 0.0 \\
    0.0 & 0.33 & 0.0 & 0.33 & 0.0 & 0.0 \\
    0.0 & 0.33 & 1.0 & 0.33 & 10.0 & 0.0 \\
    0.0 & 0.33 & 1.0 & 0.67 & 10.0 & 1.0
  \end{bmatrix}
\end{equation*}
Układ nierówności:
\begin{equation*}
  \begin{aligned}
    2.0 x_{1}+ 3.0 x_{2}+ 4.0 x_{3} &\le 6.0 \\
    2.33 x_{1}+ 3.0 x_{2}+ 4.67 x_{3} &> 7.0
  \end{aligned}
\end{equation*}
nie posiada rozwiązania, tak samo układ
\begin{equation*}
  \begin{aligned}
    2.0 x_{1}+ 3.0 x_{2}+ 4.0 x_{3} &\le 5.0 \\
    2.33 x_{1}+ 3.0 x_{2}+ 4.67 x_{3} &> 6.0
  \end{aligned}
\end{equation*}
również nie posiada rozwiązania.

Następnie wykorzystywana jest metoda programowanie dynamicznego w celu określenia czy kolejny rozkrój może polepszyć rozwiązanie. Z metody tej wynika że możliwym jest ulepszenie rozwiązania poprzez rozkrój z długości 9. Jednak układ nierówności:
\begin{equation*}
  \begin{aligned}
    2.0 x_{1}+ 3.0 x_{2}+ 4.0 x_{3} &\le 9.0 \\
    2.33 x_{1}+ 3.0 x_{2}+ 4.67 x_{3} &> 10.0
  \end{aligned}
\end{equation*}
nie posiada rozwiązania całkowitego. Wektor \mf{N} równy jest przedostatniej kolumnie ostatniej obliczonej macierzy \mf{G\tsuper{'}}, czyli $[170.0,0.0,10.0,10.0]$ oraz
\begin{equation*}
  \mf{B} =
  \begin{bmatrix}
    1.0 & -10.0 & -10.0 & -7.0 \\
    0.0 & 3.0 & 1.0 & 1.0 \\
    0.0 & 1.0 & 1.0 & 0.0 \\
    0.0 & 0.0 & 1.0 & 1.0
  \end{bmatrix}
\end{equation*}
Na podstawie \mf{N} oraz \mf{B} można uzyskać wynikowe rozkroje. Pierwszy element wektora \mf{N} okrśla że koszt następującego rozkroju wynosi $170$. Następne elementy są ilością kolejnych rozkrojów, tak więc pierwszy rozkrój nie będzie brany pod uwagę, a dwa następne zostaną wykonane $10$ razy. Macierz \mf{B} jest analizowana od drugiej kolumny. Pierwszy wiersz równy jest kosztowi długości z której ma być wykonany rozkrój pomnożonemu przez $-1$. Następnie wiersze w kolumnach określają ile elementów o danej długości powinno znaleźć się w rozkroju.

Rozwiązaniem powyższego przykładu jest układ rozkrojów:
\begin{enumerate}
  \item  elementy: $[2.0,3.0,4.0]$, ilość: 10, odpad: 0, długość bazowa: 9
  \item  elementy: $[2.0,4.0]$, ilość: 10, odpad: 0, długość bazowa: 6
\end{enumerate}
\subsubsection{Podsumowanie}
W drugiej części artykułu poświęconemu problemowi optymalnego rozkroju \cite{GilmoreGomoryV2Article}, Gilmore oraz Gomory opisali wyniki eksperymentów wykorzystujących różne warianty metody opoisanej w pierwszej części. Problem który został użyty do testów jets problemem z przemysłu papierniczego. W podstawowym zbiorze 20 problemów długości bazowe miały tę samą długość 200 in lub mnie. Ilość elementów wynikowych była z przedziału od 20 do 40. Długości elementów były od 20 in. do 80 in. z dokładnością do $1/4$ in. Liczba noży była ustawiona na pięć, siedem lub dziewięć.

Średnia ilość iteracji metody simplex dla tego problemu to w przybliżeniu 130 iteracji. Jednak rozpiętość ilości iteracji była duża od 20 do 300. Taka zmienność jest powszechna dla problemóœ programowania linowego. Problemy które są niemal identyczne mogą zachowywać się bardzo odmiennie w odniesieniu do emtdy sympleks. Zgodnie z przewidywaniami, problemy z mniejszą ilością elementów wchodzących w skład rozkroju, potrzebują mniej iteracji. Trend ten jest niedeterministyczny, dla przykładu 35 elementowy rozkrój wymagał 197 iteracji, gdzie problem pokrewny dla 40 elementów wymagał ich tylko 161.

\subsection{Metoda "Brutal Force"}\label{brutalForce}
\subsubsection{Algorytm wyjściowy}
Metoda ta opiera się zarówno na intuicji jak i na rozwiązaniu zaproponowanym przez Dantziga dla problemu plecakowego \cite{DantzigArticle}. Jest to metoda która w prosty sposób - nie używając złożonych modeli matematycznych, pozwala osiągnąć optymalny rozkrój materiału.

Pierwszym krokiem jest posortowanie elementów wyściowych malejąco wzgęldem ich długości $l_1 \ge l_2 \ge ... \ge l_m$ i umieszczenie w ten sposób w kolejce.

Drugim krokiem jest pobranie pierwszego elementu z kolejki i sprawdzenie, jak wiele razy jego długość zawiera się w długości elementu bazowego. Obliczone zostaje ile materiału pozostało w elemencie bazowym po docięciu najdłuższych elementów. Następnie pobierany jest kolejny odcinek z kolejki. Następuje sprawdzenie ile razy zawiera się on w pozostałej długości.
\begin{equation}\label{base_dantizg}
\begin{split}
& a_1 = [L/l_1],\\
& a_2 = [(L-l_1 a_1)/l_2],\\
& a_3 = [(L-(l_1 a_1+l_2 a_2))/l_3], ...
\end{split}
\end{equation}
Kroki te powtarzane są dopóki kolejka się nie skończy.

Każdy element wyjściowy posiada określoną liczebność jaką powinien osiągnąć pod koniec procesu cięcia. Jeśli na danym etapie procesu cięcia wymagana liczba elementów danego typu spada do zera, wówczas jest on pomijany w dalszej pracy algorytmu. Koniecznie jest sprawdzenie czy liczba uzyskanych elementów danego typu jest mniejsza lub równa od wymaganej:
\begin{itemize}
  \item Jeśli stwierdzenie jest prawdziwe - długość z której elementy są wycinane zostanie zmniejszona o liczbę wystąpień elementu pomnożoną przez jego długość, a licznik wymaganych odcinków danej długości zostanie zmniejszony o odpowiednią liczbę wystąpień
  \item Jeśli stwierdzenie jest fałszywe - długość z której elementy są wycinane zostanie zmniejszona o liczbę pozostałych wykrojów pomnożoną przez długość elementu, a licznik wymaganych odcinków danej długości zostanie ustawiony na zero.
\end{itemize}
Po zakończeniu przebiegu algorytmu dla jednego układu rozkroju, można określić ile razy będzie on użyty. Zostaje to wyznaczone poprzez obliczenie
\begin{equation}
g = \lfloor min\{z_i/a_i\} \rfloor, \qquad i \in [0..m], g \in Z
\end{equation}
gdzie $g$ to liczba ile razy dany schemat może zostać użyty, $z$ to liczbność wyjściowego elementu $i$ która pozostała do wycięcia, $a$ to ilość wykrojów elementu $i$ w bierzącym układzie, $m$ to liczba długości umieszczonych w rozkroju. Następnie licznik wymaganych odcinków elemntu $i$ zostaje zmniejszony o $g a_i$.

Cały proces powtarzany jest do momentu aż wszytskie wymagane elementy zostaną wycięte.

\subsubsection{Rozszerzenie o szerokość cięcia}
W warunkach rzeczywistych elementy wycinane są za pomocą ostrza które ma niezerową grubość. Wówczas metodę obliczania należy rozszerzyć jeśli ma odpowiadać warunkom rzeczywistym. Szerokość cięcia wlicza się w odpad. Jest kilka przypadków wliczania szerokości ostrza.

Jeżeli element jest równy długości bazowej wówczas nie wlicza się szerokości cięcia. Natomiast jeżeli materiał bazowy ma zostać pocięty na kilka elmentów wówczas do każdego dolicza się szerokość cięcia. Szczególnym przypadkiem jest, gdy ostatni element wraz z szerokością ostrza jest dłuższy niż długość odcinka, który został po wycięciu wcześniejszych elementów.

Gdyby szerokość cięcia nie została uwzględniona w obliczeniach wówczas dla elementu wejściowego o długości 6000mm i wymaganych odcinkach 4500mm oraz 1500mm, obie długości zostały wycięte z jednego segmentu materiału bazowego. Skutkiem takiego postępowania byłby element krótszy o szerokość ostrza. Zazwyczaj długość ta może być akceptowana jako toleracncja dokładności maszyny. Jednak dla poprawności obliczeń wielkość ta powinna zostać uwzględniona.

\subsubsection{Rozszerzenie o wiele długości bazowych}
Dla zmniejszenia odpadu można użyć kilku długości bazowych. Rozszerzenie to wprowadza następująca zmianę algorytmu: obliczenia układu muszą zostać powtórzone dla każdego elementu wejściowego. Następnie wybierany jest ten rozkrój, który daje mniejszy odpad. Modyfikacja ta znacząco wpływa na wydajność metody. Jeżeli $n$ oznacza złożoność obliczeniową podstawowego algorytmu, a $m$ oznacza liczbę odcinków wejściowych, wówczas nowa złożoność obliczeniowa wynosi $m*n$. % a może n^m

\subsubsection{Rozszerzenie o cenę materiału wsadowego}
Rozszerzenie to wprowadza zmianę koncepcyjną. Każdy element bazowy posiada cenę za metr bieżący materiału, umożliwia to obliczenie kosztu odpadu i wybranie tańszej opcji wykroju.

\subsubsection{Przykład}
\begin{enumerate}
  \item Dane wejściowe
  \begin{itemize}
    \item 6000mm - 3\$/mb
    \item 7000mm - 2\$/mb
    \item szerokość cięcia: 10mm
  \end{itemize}
  \item Dane wyjściowe
  \begin{itemize}
    \item 1x3500mm
    \item 1x3000mm
    \item 3x2000mm
    \item 5x500mm
  \end{itemize}
  \item Przebieg algorytmu
  \begin{itemize}
    \item Pierwszy rozkrój
    \begin{itemize}
      \item 3500mm mieści się raz w 6000mm. Zostaje $2500 - 10 = 2490$mm.
      \item 3000mm nie mieści się w 2490mm.
      \item 2000mm mieści się raz w 2490mm. Zostaje $490 - 10 = 480$mm.
      \item 500mm nie mieści się w 480mm.
      \item Rozkrój 6000mm: 3500mm, 2000mm. Odpad $6000 - 5500 = 500 * 0.003 = 1.5\$$
      \item ------------
      \item 3500mm mieści się dwa razy w 7000mm. Dostępny jest jeden odcinek 3500mm. Zostaje $3500 - 10 = 3490$mm.
      \item 3000mm mieści sie raz w 3490mm. Zostaje $490 - 10 = 480$mm.
      \item 2000mm nie mieści się w 480mm.
      \item 500mm nie mieści się w 480mm.
      \item Rozkrój 7000mm: 3500mm, 3000mm. Odpad $7000 - 6500 = 500 * 0.002 = 1.0\$$
      \item ------------
      \item Wybrano rozkrój 3500mm, 2000mm na długości 7000mm ze względu na mniejszy koszt odpadu.
      \item Do realizacji posostało: 0x3500mm; 0x3000mm; 3x2000mm; 5x500mm
    \end{itemize}
    \item Drugi rozkrój
    \begin{itemize}
      \item 2000mm mieści się trzy razy w 6000mm. Uwzględniając szerokość cięcia - zostaną użyte tylko dwa elementy od długości 2000mm. Zostaje $2000 - 2*10 = 1980$mm.
      \item 500mm mieści się trzy razy w 1980mm. Zostaje $480 - 3*10 = 450$mm.
      \item Rozkrój 6000mm: 2x2000mm, 3x500mm. Odpad $6000 - 5500 = 500 *0.003 = 1.5\$$
      \item ------------
      \item 2000mm mieści się trzy razy w 7000mm. Zostaje $1000 - 3*10 = 970$mm.
      \item 500mm mieści się raz w 970mm. Zostaje $470 - 10 = 460$mm.
      \item Rozkój 7000mm: 3x2000mm, 500mm. Odpad $7000 - 6500 = 500 * 0.002 = 1.0\$$
      \item ------------
      \item Wybrano rozkrój 3x2000mm, 500mm na długości 7000mm ze względu na mniejszy koszt odpadu
      \item Do realizacji posostało: 0x3500mm, 0x3000mm, 0x2000mm, 4x500mm
    \end{itemize}
    \item Trzeci rozkrój
    \begin{itemize}
      \item 500mm mieści się dwanaście razy w 6000mm. Dostępne są cztery element 500mm. Zostaje $6000 - 4*500 - 4*10 = 3960$mm.
      \item Rozkrój 6000mm: 4x500mm. Odpad $6000 - 4*500 = 4000 * 0.003 = 12\$$
      \item ------------
      \item 500mm mieści się czternaście razy w 7000mm. Dostępne są cztery elementy 500mm. zostaje $7000 - 4*500 - 4*10 = 4960$mm
      \item Rozkrój 7000mm: 4x500mm. Odpad $7000 - 4*500 = 5000 * 0.002 = 10\$$
      \item ------------
      \item Wybrano rozkrój 4x500 na długości 7000mm ze względu na mniejszy koszt odpadu
      \item Do realizacji posostało: 0x3500mm, 0x3000mm, 0x2000mm, 0x500mm
    \end{itemize}
    \item Podsumowanie
    \begin{itemize}
      \item Rozkroje : 3500mm, 2000mm na długości 7000mm; 3x2000mm, 500mm na długości 7000mm; 4x500 na długości 7000mm.
      \item Suma odpadów: $6000 * 0.002 = 12\$$
    \end{itemize}
  \end{itemize}
\end{enumerate}

\subsubsection{Podsumowanie}
Przedstawiony algorytm jest intuicyjny oraz zwraca poprawne wyniki. Główną wadą jest brak świadomości o następnym kroku oraz kolejnych wykrojach. Dla przykładu: Zosatło 1000mm materiału, do dyspozycji (z długości mniejszych niż 1000mm) jest odcinek 900mm oraz dwa elementy 480mm. Algorytm przydzieli odcinek 900mm, jednak lepszym wyborem byłoby użycie dwóch odcinków 480mm.
