\section{Zakończenie}
Problem optymalnego rozkroju rur jest szczegółowym przypadkiem problemu pleckaowego. Szczególne rozwiązania tego problemu mogą zostać osiągnięte na wiele sposobów. Porównanie dwóch algorytmów obliczania optymalnego rozkroju rur - brutalnej siły oraz opóźnionej generacji kolumn wskazuje na dwa typy metod. Obejmują one metody służące do prototypowania oraz do dokładnego obliczania wartości rozkroju. Metoda "Brutal Force" jest mniej skomplikowana oraz jest szybsza niż metoda "Delayed Column Generation", dlatego może zostać wykorzystana do szybkiego protypowania oraz przewidywania szacunkowych kosztów rozkroju. Druga metoda może zostać zastosowana do dokładnego obliczenia schematów rozkrojów. Schematy te mogą zawierać więcej elementów niż zostało zamówione lecz nadal posiadać mniejszy koszt niż metoda brutalnej siły.

Aplikacja udostępnia możliwość obliczania rozkrojów z własnych zamówień, zadanym algorytmem. Porównanie implementacji obu metod rozkroju potwierdziło, iż metoda brutalnej siły jest znacznie szybsza lecz wyniki są gorsze niż metody opóźnionej generacji kolumn. Implementacja programu wymaga wielu obliczeń macierzowych oraz wielu obliczeń wartości maksymalnej z układu nierówności. Są to operacje o bardzo dużym zapotrzebowaniu czasowym. Aplikacja może zostać rozszerzona o obsługę przypadku gdy w trakcie metody opóźnionej generacji wystepuje ujemna wartość kolumny liczebności danego rozkroju. Obecnie gdy taka sytuacja wystąpi zwracany jest ostatni znany poprawny rozkrój. W trakcie przeprowadzania eksperymentu, wyniki te zostały odrzucone ze względu na możliwość, iż sytuacja ta spowodowana jest losowymi danymi, które mogły mieć nieprawidłowy format wejściowy, lub ze względu iż przypadek ten został pominięty w implementacji. Kolejnym usprawnieniem aplikacji może zostać podzielenie obliczeń na wątki, tak aby praca został zrównoleglona oraz przyspieszona.

Rozszerzeniem algorytmu zastosowanego w metodzie opóźninej generacji kolumn może zostać metoda medianowa, zaproponowana przez Gilmorea oraz Gomorego \cite{GilmoreGomoryV2Article}. Metoda ta według przeprowadzonych eksperymentów może skrócić czas oraz obiżyć zapotrzebowanie na zasoby obliczeniowe nawet o 90\%.
